\documentclass[12pt]{article}

% --- Paquetes básicos ---
\usepackage[utf8]{inputenc}
\usepackage[spanish]{babel}
\usepackage{amsmath}
\usepackage{amsfonts}
\usepackage{graphicx}
\usepackage{geometry}

% --- Configuración de márgenes ---
\geometry{margin=1in}

% --- Información del documento ---
\title{Ensayo: La Programación en el estudio de la Física Atómica}
\author{Pedro Leonardo Pop Gamboa}
\date{\today}

\begin{document}

\maketitle

\section*{Mi área de investigación: Física Atómica}
Mi área de interés se centra en la \textbf{Física Atómica y Molecular}. Este campo es fundamental para comprender las interacciones entre la materia y la energía a escalas fundamentales. Se encarga de estudiar la estructura electrónica de los átomos, los niveles de energía y cómo estos sistemas interactúan con campos electromagnéticos, lo cual es la base de tecnologías modernas como los relojes atómicos y los láseres de precisión, este es el tema que mas me interesa de las propuestas que tenia para esta tarea :p

\section*{La programación como herramienta}
En la actualidad, es imposible concebir la física atómica sin el apoyo de la programación y el cálculo computacional. El uso de algoritmos y simulaciones facilita el trabajo científico por las siguientes razones:

\begin{itemize}
    \item \textbf{Modelado de Sistemas Multielectrónicos:} Aunque la ecuación de Schrödinger puede resolverse exactamente para el átomo de hidrógeno, los átomos con más electrones requieren métodos numéricos complejos como \textit{Hartree-Fock} o la \textit{Teoría del Funcional de la Densidad} (DFT). Estos cálculos dependen totalmente de algoritmos optimizados.
    
    \item \textbf{Análisis de Espectroscopía:} La identificación de elementos a través de la luz que emiten genera grandes volúmenes de datos. Mediante la programación, podemos automatizar la comparación de espectros experimentales con bases de datos teóricas de manera eficiente.
    
    \item \textbf{Simulación de Trampas de Iones:} El diseño de experimentos que utilizan láseres para enfriar átomos requiere simulaciones previas para predecir el comportamiento de las partículas en condiciones de vacío extremo, lo cual ahorra recursos y tiempo en el laboratorio.
\end{itemize}

\section*{Conclusión}
La programación no es solo una herramienta auxiliar, sino un lenguaje adicional para el físico. En la física atómica, nos permite transformar ecuaciones teóricas abstractas en predicciones tangibles y controlar experimentos con una precisión que sería inalcanzable de forma manual.

\end{document}